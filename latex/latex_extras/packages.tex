%These are the packages that we will use in our latex document. 
\usepackage{clipboard}%For copy past text in different parts of document
\usepackage{amsmath, amssymb,mathrsfs} %For using math
\usepackage{color}
\usepackage{enumitem}
\usepackage{graphicx}
\usepackage{grffile}
\usepackage[utf8]{inputenx}% For proper input encoding
\usepackage{mathptmx}
\usepackage{pdflscape}
\usepackage{soul}
%\usepackage{libertine}% Linux Libertine, may favourite text font
\usepackage[euler-digits]{eulervm}% A pretty math font

%\usepackage{fontspec}
%\usepackage{unicode-math}

%\setmainfont{Linux Libertine O}

%\newcommand{\setlibertinemath}{%
% use Libertine for the letters
%\setmathfont[range=\mathit/{latin,Latin,num,Greek,greek}]{Linux Libertine O Italic}
%\setmathfont[range=\mathup/{latin,Latin,num,Greek,greek}]{Linux Libertine O}
%\setmathfont[range=\mathbfup/{latin,Latin,num,Greek,greek}]{Linux Libertine O Bold}
%\setmathfont[range={"2202}]{Linux Libertine O}% "02202 = \partial % doesn't work
%\setmathfont[range={"221E}]{Linux Libertine O}% "0221E = \infty
% etc. (list should be completed depending on needs)
%}



%%% ABSTRACT
\usepackage{abstract}
\renewcommand{\abstractname}{}    % clear the title
\renewcommand{\absnamepos}{empty} % originally center

%%% PAGE DIMENSIONS
\usepackage[margin=1in]{geometry} %For controlling the dimensions. Very useful for Posters, etc
% PAPER SIZE
\setlength\paperheight{11in}
\setlength\paperwidth{8.5in}
% MARGINS 
%\topmargin=-0.5in
\parskip=0pt
%\setlength\textwidth{135.5mm}
%\setlength\textheight{200mm}
%\setlength\topmargin{-10pt}
%\setlength\oddsidemargin{14mm}
%\setlength\evensidemargin{\oddsidemargin}
%\setlength\headheight{24pt}
%\setlength\headsep{14pt}
%\setlength\topskip{11.74pt}
%\setlength\maxdepth{.5\topskip}
%\setlength\footskip{15pt}

%%% HEADERS AND FOOTERS
\usepackage{fancyhdr}
%\pagestyle{plain} % options: empty , plain , fancy
%\renewcommand{\headrulewidth}{0pt} % customise the layout...
%\lhead{}\chead{}\rhead{}
%\lfoot{}\cfoot{\thepage}\rfoot{}

%%% SECTION TITLE APPEARANCE
%\usepackage{sectsty}
%%\sectionfont{\large}
%\sectionfont{\normalsize}
%\subsectionfont{\small}
%\paragraphfont{\small}
%\allsectionsfont{\mdseries\upshape 
%        \sectionrule{15pt}{0pt}{-5pt}{1pt} }
\usepackage{titlesec}
\titleformat*{\section}{\normalsize\bfseries}
\titleformat*{\subsection}{\small\bfseries}
\titleformat*{\subsubsection}{\small\bfseries}
\titleformat*{\paragraph}{\small\bfseries}
\titleformat*{\subparagraph}{\small\bfseries}
\titlespacing\section{0pt}{6pt plus 4pt minus 2pt}{0pt plus 2pt minus 2pt}
\titlespacing\subsection{0pt}{6pt plus 4pt minus 2pt}{0pt plus 2pt minus 2pt}
\titlespacing\subsubsection{0pt}{6pt plus 4pt minus 2pt}{0pt plus 2pt minus 2pt}
\titlespacing\paragraph{0pt}{6pt plus 4pt minus 2pt}{6pt plus 2pt minus 2pt}

%%% CITATIONS
\usepackage[authoryear]{natbib}	%Bibliography package. It handles citations and can easily change formats. This is done at the end of the document
\usepackage{multibib}
\newcites{App}{Appendix References}%  \citelatex, \nocitelatex, ...
\newcites{Ref}{References in Referee Replies}%  \citelatex, \nocitelatex, ...
\newcites{Main}{References}%  \citelatex, \nocitelatex, ...

\setcitestyle{aysep={}} 
%------------------------------------------%
%     Tables 
%------------------------------------------%
\usepackage{accents}
\usepackage{adjustbox} %Handles resizing of tables, figures, etc. Based off of page and/or line width/height 
\usepackage{bbm}
\usepackage{bm}
\usepackage{booktabs}% For Pretty tables
\usepackage{caption} %\captionsetup[figure]{justification=raggedright,singlelinecheck=off}
\captionsetup[table]{labelsep=period}
\captionsetup{labelsep=period}
\usepackage{chngpage} 
\usepackage{comment}
\usepackage{dcolumn}
%\usepackage{everypage}%For page numbers on rotated pages%Functionality added to latex
%\usepackage{float}
\usepackage[capposition=top]{floatrow}
\usepackage{morefloats}
\usepackage{multicol}
\usepackage{multirow}
\usepackage{placeins}% To Create Float Barriers so the tables will stay in their sections
\usepackage[list=true]{subcaption} %For having multiple figures within the same one. Figure 1, with part (a) and (b) %\setcounter{lofdepth}{2}
\usepackage{threeparttable}% For Notes below table
\usepackage{rotating}% To Rotate Table
\usepackage{setspace} %Single, Double Space, etc 
\usepackage{siunitx} %This aligns tables by their decimal and handles processing of the numbers within the tables
  \sisetup{
    detect-mode,
    group-digits      = false,
    input-symbols     = {( ) [ ] - +},
    table-align-text-post = false,
    input-signs             = ,
    %parse-numbers=false,
    %scientific-notation = true,
        %round-mode              = places,
        %round-precision         = 2,
    %input-ignore={,},
    input-decimal-markers={.},
    group-separator={,},
    %group-minimum-digits={4},
    %group-four-digits = true,
	separate-uncertainty=true,
	%zero-decimal-to-integer=true,
   } 
%------------------------------------------%
%     Formatting options 
%------------------------------------------%

\AtBeginDocument{
    \addtolength{\abovedisplayskip}{-1.0ex}
    \addtolength{\abovedisplayshortskip}{-1.0ex}
    \addtolength{\belowdisplayskip}{-1.0ex}
    \addtolength{\belowdisplayshortskip}{-1.0ex}
}
\setlength{\parskip}{\smallskipamount}%med
%\setlength{\parindent}{0pt}
\DeclareCaptionLabelFormat{blank}{}
\DeclareCaptionFormat{myformat}{
    \begin{varwidth}{\linewidth}
        \centering
        #1#2#3%
    \end{varwidth}
}
\usepackage{silence}
\WarningFilter*{latex}{Text page \thepage\space contains only floats}
\WarningFilter*{caption}{Ambiguous sub-caption}
\WarningFilter*{natbib}{Citation}
\WarningFilter{natbib}{Citation}

%------------------------------------------%
%     Hyper Ref 
%------------------------------------------%
%\usepackage{silence}
%\WarningFilter{hyperref}{You have enabled option `breaklinks'.}
\usepackage[unicode=true,pdfusetitle,bookmarks=true,bookmarksnumbered=false,bookmarksopen=false,breaklinks=true,pdfborder={0 0 1},backref=false,colorlinks=true,citecolor=black,hyperfootnotes=true]{hyperref}
\def\UrlBreaks{\do\/\do-\do.\do=\do_\do?\do\&\do\%\do\a\do\b\do\c\do\d\do\e\do\f\do\g\do\h\do\i\do\j\do\k\do\l\do\m\do\n\do\o\do\p\do\q\do\r\do\s\do\t\do\u\do\v\do\w\do\x\do\y\do\z\do\A\do\B\do\C\do\D\do\E\do\F\do\G\do\H\do\I\do\J\do\K\do\L\do\M\do\N\do\O\do\P\do\Q\do\R\do\S\do\T\do\U\do\V\do\W\do\X\do\Y\do\Z\do\0\do\1\do\2\do\3\do\4\do\5\do\6\do\7\do\8\do\9} 
%\usepackage[hyphenbreaks]{breakurl}
% hyperef does not work with beamer (as far as I know) so comment this out if you are using it. 
% Also leave the hyperfootnotes as false as when they are true they oddly clash with the caption or subcaption package.

%------------------------------------------%
%    Appendix
%------------------------------------------%
\usepackage[titletoc]{appendix}

%------------------------------------------%
%    To Do Notes
%------------------------------------------%

%\usepackage[colorinlistoftodos,prependcaption,textsize=tiny]{todonotes}
\usepackage{xcolor}
\usepackage{xargs}
%Use \todo to create a general todo item for anyone
%Something for Chris to do
%\newcommandx{\todochris}[2][1=]{\todo[linecolor=green,backgroundcolor=green!25,bordercolor=yellow,#1]{Chris: #2}}
%\newcommandx{\todokk}[2][1=]{\todo[linecolor=green,backgroundcolor=green!25,bordercolor=yellow,#1]{Chris: #2}}
%Something for Anthony to do
%\newcommandx{\todoanthony}[2][1=]{\todo[linecolor=green,backgroundcolor=green!25,bordercolor=yellow,#1]{Anthony: #2}}
% To just make a comment use this
%\newcommandx{\todocomment}[2][1=]{\todo[linecolor=yellow,backgroundcolor=yellow!25,bordercolor=cyan,#1]{Comment: #2}}
\renewcommand{\thefootnote}{\fnsymbol{footnote}}
